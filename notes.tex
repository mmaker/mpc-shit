\documentclass[10pt,a4paper,oneside]{memoir}

\usepackage[utf8]{inputenc}
\usepackage{amsmath,amsfonts,amssymb,amsthm}
\usepackage{enumitem}
\usepackage{braket}

\newtheorem{thm}{Theorem}[chapter]
\theoremstyle{definition}
\newtheorem{dfn}[thm]{Definition}

\begin{document}

\begin{dfn}[Perfect Correctedness]
  All players receive outputs that are correct based on the input supplied.
\end{dfn}

\begin{dfn}[Perfect Privacy]
  Any subset $C$ of $t < n/2$ corrupted players gains no information beyond the
  pairs $\set{x_j, y_j}_{P_j \in C}$ of inputs, outputs of the corrupted
  players, regardless of their computing power.
\end{dfn}

\begin{dfn}[Simulation Paradigm]
  Technique to prove that someone learns only information $X$ by showing that
  everything he sees can be \emph{efficiently} recreated, or simulated, given
  only $X$.
\end{dfn}

\begin{dfn}[Robust Protocol]
  A protocol is said to be \emph{robust} if correct and private, even if some
  parties do not follow the protocol.
\end{dfn}

\end{document}
